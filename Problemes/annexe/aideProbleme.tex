
%%%%%%%%%%%%%%%%%%%%%
\chapter{Aide aux problèmes}
%%%%%%%%%%%%%%%%%%%%%

%%%%%%%%%%%%%%%%%%%%%
\section{Strategix}
%%%%%%%%%%%%%%%%%%%%%

\subsection{25}

Analyse de la position : Les pions doublées donnent deux colonnes semi-ouvertes au noirs (les blancs n'en ont qu'une).
Le pion a5 des noir est faible.
Le fou des noirs est un mauvais fou (les pions des noirs sont sur des cases noires) et il est inactif. Le cavalier des blanc dispose de la case e4 et rêve de la case e6

\begin{minipage}{0.45\textwidth}
\hspace{0.7cm}
\vspace{0.15cm}

\hspace{0.7cm}

\vspace{0.15cm}
\hspace{0.7cm} Un bon coup ici est d'échanger du matériel contre de l'espace.

\vspace{0.15cm}
\hspace{0.7cm}Ou se trouve les cases faibles des noirs ? Comment y placer des pièces ?

%\vspace{0.5cm}
%\hspace{0.7cm}{\footnotesize source : youtube, chesstrainer2000.}
%\begin{itemize}[leftmargin=0.7cm, itemsep=0pt]
%\item  \end{itemize}
\end{minipage}
\hfill
\begin{minipage}{0.45\textwidth}
\newgame
\restoregame{strategie25}
\chessboard %\showboard
\end{minipage}


%%%%%%%%%%%%%%%%%%%%%%%%%%%%%%%%%%%%%%%%%%%%%%%%%%%%%%%%%

%\begin{itemize}[leftmargin=1cm, label=\ding{32}, itemsep=1pt]
%\item {\bf } : \end{itemize}

%%%%%%%%%%%%%%%%%%%%%%%%%%%%%%%%%%%%%%%%%%%%%%%%%%%%%%%
