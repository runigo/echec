

%\input{./utilisation/exemple.tex}

\section{Principes élémentaires}
%
%%%%%%%%%%%%%%%%%%%%%
\subsection{Prendre le centre}
%%%%%%%%%%%%%%%%%%%%%
%
L'occupation du centre consiste à placer une pièce au centre. Cela peut être un pion ou une pièce mineure. Contrôler le centre consiste à placer une pièce qui attaque une ou plusieurs case du centre.

\begin{center}
\newgame
\mainline{1. e4 }
\def\empharea{ e4-e4 }
\chessboard[color=red,
	markstyle=color,markfields=d5,
	emphstyle=\color{green},
	empharea=\empharea]
\end{center}

Le pion e4 occupe une case centrale et contrôle la case centrale (d5). C'est un bon coup.

\begin{center}
\newgame
\mainline{1. Nf3 }
\def\empharea{ f3-f3 }
\chessboard[color=red,
	markstyle=color,markfields=d4,
	markstyle=color,markfields=e5,
	emphstyle=\color{green},
	empharea=\empharea]
\end{center}

Le cavalier f3 contrôle les deux cases centrales d4 et e5.

\newgame

\mainline{1. e4 e5 2. Nf3 Nc6}

\begin{center}
\chessboard
\end{center}

%\newchessgame
\newgame

\def\empharea{ h8-f4 }
\chessboard[emphstyle=\color{red},
empharea=\empharea]



%%%%%%%%%%%%%%%%%%%%%
\subsection{Se développer}
%%%%%%%%%%%%%%%%%%%%%
%

\subsubsection{Sortir les pièces mineures}


\subsubsection{Gagner du temps}

Créer une menace en se développant oblige l'adversaire à contrer cette menace. 



%%%%%%%%%%%%%%%%%%%%%
\subsection{Roquer}
%%%%%%%%%%%%%%%%%%%%%
%

%%%%%%%%%%%%%%%%%%%%%%%%%%%%%%%%%%%%%%%%%%%%%%%%%%%%%%%%%%%%%%%%%%%%%%%%


%%%%%%%%%%%%%%%%%%%%%%%%%%%%%%%%%%%%%%%%%%%%%%%%%%%%%%%%%%%%%%%%%%%%%%%%
