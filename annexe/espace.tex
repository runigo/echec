
%%%%%%%%%%%%%%%%%%%%%
\chapter{L'espace}
%%%%%%%%%%%%%%%%%%%%%

%%%%%%%%%%%%%%%%%%%%%
\section{Espace contrôlé}
%%%%%%%%%%%%%%%%%%%%%

%C'est les cases accessible par les pièces (pas par les pions)

Lors de 

\begin{minipage}{0.4\textwidth}

\begin{center}
\newgame
\mainline{1. e4 }

\chessboard[color=red,
	markstyle=color,markfields=a6,
	markfields=e2,markfields=d3,markfields=c4,markfields=b5,
	markfields=f3,markfields=g4,markfields=h5,]
\end{center}
L'ouverture du pion roi offre 8 cases.
\end{minipage}
\begin{minipage}{0.5\textwidth}
\begin{center}
\newgame
\mainline{1. d4 }

\chessboard[color=red,
	markstyle=color,markfields=h6,
	markfields=d2,markfields=e3,markfields=f4,markfields=g5,
	markfields=d3,]
\end{center}
L'ouverture du pion roi offre 6 cases.
\end{minipage}

\begin{minipage}{0.4\textwidth}
\begin{center}
\newgame
\mainline{1. c4 }

\chessboard[color=red,
	markstyle=color,markfields=d4,
	markstyle=color,markfields=e5,]
\end{center}
Le cavalier f3 contrôle les deux cases centrales d4 et e5.
\end{minipage}
\begin{minipage}{0.4\textwidth}
\begin{center}
\newgame
\mainline{1. Nf3 }

\chessboard[color=red,
	markstyle=color,markfields=d4,
	markstyle=color,markfields=e5,]
\end{center}
Le cavalier f3 contrôle les deux cases centrales d4 et e5.
\end{minipage}

\newgame

\mainline{1. e4 e5 2. Nf3 Nc6}

\begin{center}
\chessboard
\end{center}

%\newchessgame
\newgame

\def\empharea{ h8-f4 }
\chessboard[emphstyle=\color{red},
empharea=\empharea]


%%%%%%%%%%%%%%%%%%%%%
%\section{Le temps}
%%%%%%%%%%%%%%%%%%%%%

\fenboard{r5k1/1b1p1ppp/p7/1p1Q4/2p1r3/PP4Pq/BBP2b1P/R4R1K w − − 0 20}
\mbox{}
\bigskip
\showboard
\mainline{20.Qxb7 Rae8 21.Qd5}





%\newgame
%\mainline{1. Nf3 }
%\def\empharea{ f3-f3 }
%\chessboard[color=red,
%	markstyle=color,markfields=d4,
%	markstyle=color,markfields=e5,
%	emphstyle=\color{green},
%	empharea=\empharea]
%%%%%%%%%%%%%%%%%%%%%%%%%%%%%%%%%%%%%%%%%%%%%%%%%%%%%%%
