
\newpage
%%%%%%%%%%%%%%%%%%%%%
\section{Solutions}
%%%%%%%%%%%%%%%%%%%%%

\subsection{Problème 1}%%%%%%%%%%%%%%%%%%%%%%%%%%%%%%%%%%%%%%%%%%%%
\newgame
\restoregame{probleme01}
\begin{minipage}{0.45\textwidth}
\hspace{0.7cm} Le bon coup est d'activer le fou grâce au sacrifice %\mainline{1... e4}
\vspace{0.25cm}

\hspace{0.7cm} En contrepartie, le cavalier blanc arrive sur la bonne case e4 en gagnant un pion.
\vspace{0.25cm}

\hspace{0.7cm} L'activation du mauvais fou des noirs va lui permettre de rayonner sur les cases noires dans le camps des blancs. %\mainline{2. Ne4 Bd4}
\vspace{0.25cm}

\hspace{0.7cm} Le plan des noir est d'attaquer sur la colonne b.
\end{minipage}
\hfill
\begin{minipage}{0.45\textwidth}
\chessboard[
%showmover=false,
inverse,markstyle=leftborder,
]
\end{minipage}

\begin{minipage}{0.45\textwidth}
\hspace{0.7cm} %\mainline{3. Kh1 Rfb8}% 4. b3 a4
\vspace{0.25cm}
\end{minipage}
\hfill
\begin{minipage}{0.45\textwidth}
\chessboard[
%showmover=false,
inverse,markstyle=leftborder,
]
\end{minipage}

\subsection{Exercices 28} %%%%%%%%%%%%%%%%%%%%%%%%%%%%%%%%%%%%%%%%%%%%
\subsubsection{Exercice a} %%%%%%%%%%%%%%%%%%%%%%%%%%%%%%%%%%%%%%%%%%%%
\newgame
\restoregame{probleme01}
\begin{minipage}{0.45\textwidth}
\hspace{0.7cm} fffffffffffffffff %\mainline{1... Be7} afin de conserver une bonne structure de pions.

\chessboard[
inverse,markstyle=leftborder,
arrow=latex,
pgfstyle=straightmove,
shortenstart=0.1em,
%color=blue,
opacity=0.6,
color=green,
linewidth=3pt,
markmoves={b4-e7},
]

\end{minipage}
\hfill
\begin{minipage}{0.45\textwidth}
\newgame
\restoregame{probleme01}
\hspace{0.7cm} xxxxxxxxxxxxx %\mainline{1... Nbd7}.

\chessboard[
inverse,markstyle=leftborder,
arrow=latex,
pgfstyle=straightmove,
shortenstart=0.1em,
%color=blue,
opacity=0.6,
color=green,
linewidth=3pt,
markmoves={b8-d7},
]

\hspace{0.7cm} xxxxxxxxxxxxxxxxxxxx
\end{minipage}

\subsection{Exercice 33} %%%%%%%%%%%%%%%%%%%%%%%%%%%%%%%%%%%%%%%%%%%%

\newgame
\restoregame{strategie33}
\begin{minipage}{0.45\textwidth}
\hspace{0.7cm} Le bon coup est \mainline{1. g4}. Le cavalier des noirs n'a plus d'avenir (le meilleur coup pour les noirs est alors de le ramener en g8).
\vspace{0.5cm}

\hspace{0.7cm} Les blancs conserve la paire de fous. Leur bon fous (celui de case noir) possède de bonnes cases pour se développer

\end{minipage}
\hfill
\begin{minipage}{0.45\textwidth}
\chessboard
\end{minipage}

\begin{minipage}{0.45\textwidth}
\hspace{0.7cm} Si les noirs jouent le petit roque, \mainline{1... O-O}, les blancs ont une attaque après \mainline{2. h4}.
\vspace{0.5cm}

\hspace{0.7cm} Les blancs ont un avantage d'espace et le plan fou g5 et dame d2 (attaque le cavalier et donne la possibilité du grand roque).
\vspace{0.5cm}

\end{minipage}
\hfill
\begin{minipage}{0.45\textwidth}
\chessboard[arrow=latex,
pgfstyle=straightmove,
shortenstart=0.1em,
color=blue,
linewidth=3pt,
markmoves={d1-d2},
markmoves={c1-g5}
]
\end{minipage}
%%%%%%%%%%%%%%%%%%%%%%%%%%%%%%%%%%%%%%%%%%%%%%%%%%%%%%%
%\begin{itemize}[leftmargin=1cm, label=\ding{32}, itemsep=1pt]
%\item {\bf } : \end{itemize}
