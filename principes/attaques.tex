

%\input{./utilisation/exemple.tex}

\section{Attaques}
%
Pour des parties offensives
%%%%%%%%%%%%%%%%%%%%%
\subsection{Roques symétriques}
%%%%%%%%%%%%%%%%%%%%%
%
Attaquer suivant la direction de ses pions,

Si on se fait attaquer à l'aile, contre-attaquer au centre.

Si on se fait attaquer au centre, contre-attaquer à l'aile.
%%%%%%%%%%%%%%%%%%%%%
\subsection{Roques opposés}
%%%%%%%%%%%%%%%%%%%%%
%
Favorise les parties d'attaques

Le grand roque est interressant lorsqu'une attaque se dessine sur le petit roque adverse.

Attaquer attaquer le roque adverse


Lancer les pions



Ne pas défendre son roque, ne pas ouvrir son roi, sacrifier des pièces, jouer pour le mat


%%%%%%%%%%%%%%%%%%%%%%%%%%%%%%%%%%%%%%%%%%%%%%%%%%%%%%%%%%%%%%%%%%%%%%%%
