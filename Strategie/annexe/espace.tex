
%%%%%%%%%%%%%%%%%%%%%
\chapter{L'espace}
%%%%%%%%%%%%%%%%%%%%%

%L'avance des pions offre de l'espace.

\section{Exemple}
%%%%%%%%%%%%%%%%%%%%%%%%%%%%%
\newgame
\begin{minipage}{0.45\textwidth}

%Dans l'ouverture du cavalier du roi, \mainline{1. e4 e5 2. Nf3 } les noir peuvent défendre par \mainline{2... d6}.
%Les blancs peuvent alors poursuivre la prise du centre, \mainline{3.d4 }
%et si les noirs prennent \mainline{3... exd4 },
%Les blancs prennent un avantage par \mainline{4. Qxd4 Nc6}
\mainline{1. e4 e5 2. Nf3 d6 3.d4 }

\hspace{1.7cm}\mainline{3... exd4 4. Qxd4 Nc6}

\hspace{0.7cm}Les blancs ont plus d'espace : ils occupent quatre traverses, les noirs trois seulement. Leur fous dispose de nombreuses cases pour se développer, le fou noir des noirs est bloqué par un pion.

\hspace{0.7cm}Le coup exd4 a été l'erreur des noirs qui a permis au blanc de prendre un avantage d'espace.
\end{minipage}
\hfill
\begin{minipage}{0.45\textwidth}
\chessboard
\end{minipage}


%%%%%%%%%%%%%%%%%%%%%
\section{Premier coup}
%%%%%%%%%%%%%%%%%%%%%

%C'est les cases accessible par les pièces (pas par les pions)



\begin{minipage}{0.45\textwidth}
\hspace{0.7cm}Avant de jouer le premier coup, les blancs contrôlent les trois premières traverses,
les noirs contrôlent les trois dernières.
%En particulier, les cases f2 et f7 ne sont contrôlées que par les roi.
%Ce sont les cases les plus faibles de chacun des camps.

\vspace{0.5cm}
\hspace{0.7cm}Les traverses 4 et 5 ne sont contrôlé par aucun des camps et le premier coup joué
part à l'assaut de ces cases : {\bf Les coups de pions permettent de gagner de l'espace}.

\end{minipage}
\hfill
\begin{minipage}{0.45\textwidth}
\newgame
\def\empharea{ a6-h8 }
\chessboard[emphstyle=\color{red},
	empharea = a1-h3,
	emphstyle=\color{blue},
	empharea=\empharea ]
\end{minipage}



\begin{minipage}{0.45\textwidth}
%\begin{center}
\newgame
\mainline{1. e4 }

\chessboard[color=red,
	markstyle=color,markfields=a6,
	markfields=d5,markfields=c4,markfields=b5,
	markfields=f5,markfields=g4,markfields=h5,]
%\end{center}

L'ouverture du pion roi permet de
contrôler 7 nouvelles cases.
\end{minipage}
\hfill
\begin{minipage}{0.45\textwidth}
%\begin{center}
\newgame
\mainline{1. d4 }

\def\empharea{ d4-d4 }
\chessboard[color=red,
	markstyle=color,markfields=h6,
	markfields=c5,markfields=f4,markfields=g5,
	markfields=e5,emphstyle=\color{red},
	empharea=\empharea ]
%\end{center}

L'ouverture du pion dame permet de
contrôler 6 nouvelles cases.
\end{minipage}

%%%%%%%%%%%%%%%%%%%%%%%%%%%%%%%%%%%%%%%%%%%%%%%%%%%%%%%%%%%%%%%%%%%%

\vspace{0.5cm}

\begin{minipage}{0.45\textwidth}
\newgame
\mainline{1. Nf3 }

\chessboard[color=red,
	markstyle=color,markfields=d4,
	markstyle=color,markfields=e5,]

Le cavalier f3 contrôle les deux cases centrales d4 et e5.
\end{minipage}
\hfill
\begin{minipage}{0.45\textwidth}
\newgame
\mainline{1. c4 }

\chessboard[color=red,
	markstyle=color,markfields=d5,
	markfields=b5,markfields=a4,]

Le pion c4 contrôle la case centrale d5
et la case b5. La dame contrôle la case a4.
\end{minipage}

%\newchessgame
%\newgame
%\def\empharea{ h8-f4 }
%\chessboard[emphstyle=\color{red},
%empharea=\empharea]


%%%%%%%%%%%%%%%%%%%%%
%\section{Le temps}
%%%%%%%%%%%%%%%%%%%%%

%\fenboard{r5k1/1b1p1ppp/p7/1p1Q4/2p1r3/PP4Pq/BBP2b1P/R4R1K w − − 0 20}
%\mbox{}
%\bigskip
%\showboard
%\mainline{20.Qxb7 Rae8 21.Qd5}

%\newgame
%\mainline{1. Nf3 }
%\def\empharea{ f3-f3 }
%\chessboard[color=red,
%	markstyle=color,markfields=d4,
%	markstyle=color,markfields=e5,
%	emphstyle=\color{green},
%	empharea=\empharea]
%%%%%%%%%%%%%%%%%%%%%%%%%%%%%%%%%%%%%%%%%%%%%%%%%%%%%%%
