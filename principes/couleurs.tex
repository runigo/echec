

%\input{./utilisation/exemple.tex}

\section{Fous et jeu de couleur}
%
Les pièces mineures valent environ 3 pions.
L'ouverture conduit généralement à des positions où des échanges de pièces mineures sont possible.
Ces échanges sont parfois avantageux.

%%%%%%%%%%%%%%%%%%%%%
\subsection{Bon fou / mauvais fou}
%%%%%%%%%%%%%%%%%%%%%
%
Lorsque la structure des pions centraux est fixé et qu'il sont sur une certaine couleur, notre bon fou est celui de couleur opposé. En effet, nos pions ne lui enlève pas d'espace. L'autre fou est notre mauvais fou car nos propre pions centraux lui enlève de l'espace.
\subsubsection{fou actif/ fou passif}
Si le mauvais fou est devant sa structure de pion, il est actif, il tape dans le camp adverse, il peut être plus fort que le bon fou.
%%%%%%%%%%%%%%%%%%%%%
\subsection{fous de couleur opposée}
%%%%%%%%%%%%%%%%%%%%%
Après des échanges chaque joueur ne possède plus qu'un fous, et  que ces deux fous sont de couleur opposé.

Alors,

\begin{itemize}[leftmargin=1.7cm, label=\ding{32}, itemsep=0pt]%\end{itemize}
\item  {\bf placer ses pions sur la couleur du fous adverse} : enlève de l'espace au fou adverse, donne de l'espace à notre fou, très bon stratégiquement si nos pions sont difficilement attacable ou facilement défendable.
\item  {\bf placer ses pions sur la couleurs de notre fous} : nos pions ne peuvent plus être attaquer par le fou adverse, bon stratégiquement si nos pions étaient facilement attacable, difficilement défendable et si notre fou est actif.
\end{itemize}




%%%%%%%%%%%%%%%%%%%%%
\subsection{Paire de fous}
%%%%%%%%%%%%%%%%%%%%%
%%%%%%%%%%%%%%%%%%%%%
\subsubsection{Qui possède les fous possède l'avenir}
%%%%%%%%%%%%%%%%%%%%%
Lorsqu'on envisage l'échange de pièces mineurs, conserver la paire de fou procure généralement un avantage stratégique à long terme. En début de partie ou lorsque la position est fermée, un cavalier est plus fort qu'un fou. Lorsque la position est ouverte et en fin de partie, un fou est plus fort qu'un cavalier. 
%
%%%%%%%%%%%%%%%%%%%%%%%%%%%%%%%%%%%%%%%%%%%%%%%%%%%%%%%%%%%%%%%%%%%%%%%%
%Parties d-échecs pédagogiques avec progression 🎓 1300 elo.mp4
% Capture du 2022-03-13 21-18-02.png : 35 minute
%  Protéger avec le fou permet de se débarasser du problème de son développement

%Parties d-échecs pédagogiques avec progression 🎓 1300 elo.mp4
%Capture du 2022-03-13 22-04-16.png : 1 h 11 min
%  Les pions adverses sont sur case blanche, l'adversaire est donc faible sur case noir, échanger le fou de case blanche contre un cavalier va donner un avantage matériel sur les cases noires.

%Capture du 2022-03-13 22-21-45.png, 1 h 20 min
% Pour augmenter l'avantage sur les cases noires, Te5?!


%PSG-ed.mp4, 16 min
%Capture du 2022-03-13 23-05-03.png
%finale égale 
%%%%%%%%%%%%%%%%%%%%%%%%%%%%%%%%%%%%%%%%%%%%%%%%%%%%%%%%%%%%%%%%%%%%%%%
