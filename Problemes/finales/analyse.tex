
\newpage
%%%%%%%%%%%%%%%%%%%%%
%\chapter{Analyse}
%%%%%%%%%%%%%%%%%%%%%
\section{Analyse des position}

\subsection{Problème 1}%%%%%%%%%%%%%%%%%%%%%%%%%%%%%%%%%%%%%%%%%%%%

\newgame
\restoregame{probleme01}
\begin{minipage}{0.45\textwidth}
\hspace{0.7cm} Le matériel est égale, le pion passé des blancs est plus avancé et il on le trait. Leur position est meilleure.

\hspace{0.7cm} Après \mainline{1. c6 a4}, les noirs ne peuvent plus empêcher la poussée c7. 

\chessboard[pgfstyle=color,
opacity=0.3,
color=green,
markfield=c5,markfield=c6,
color=red,
markfield=a5,markfield=a4,
arrow=latex,
pgfstyle=straightmove,
shortenstart=0.1em,
linewidth=3pt,
color=blue,
markmoves={c6-c7},
]
\end{minipage}
\hfill
\begin{minipage}{0.45\textwidth}
\newgame
\restoregame{probleme01}
\hspace{0.7cm} Les noirs doivent empêcher la promotion en jouant \mainline{1. c6 Rd8}

\hspace{0.7cm} 
\chessboard[pgfstyle=color,
opacity=0.3,
color=green,
markfield=c5,markfield=c6,
color=red,
markfield=d4,markfield=d8,
arrow=latex,
pgfstyle=straightmove,
shortenstart=0.1em,
linewidth=3pt,
color=blue,
markmoves={c6-c7},
]
 
\end{minipage}

\subsection{Exercices 28} %%%%%%%%%%%%%%%%%%%%%%%%%%%%%%%%%%%%%%%%%%%%
\subsubsection{Exercice a} %%%%%%%%%%%%%%%%%%%%%%%%%%%%%%%%%%%%%%%%%%%%

\begin{minipage}{0.45\textwidth}
\newgame
\restoregame{probleme01}
\chessboard[
inverse,markstyle=leftborder,
%color=blue,
opacity=0.3,
color=green,
markmoves={d8-f6},
markmoves={f3-f6},
markmoves={g7-f6},
]

\hspace{0.7cm} Les blancs menacent de prendre deux fois en f6 pour détruire la structure des pions de l'aile roi.
\end{minipage}
\hfill
\begin{minipage}{0.45\textwidth}
\vspace{0.15cm}

\hspace{0.7cm} Après %\mainline{1... h6 2.Bxf6 Qxf6 3.Qxf6 gxf6}

\chessboard[
inverse,markstyle=leftborder,
pgfstyle=color,
opacity=0.15,
color=red,
markfield=f7,
markfield=f6,
markregion={g3-g8},
arrow=latex,
pgfstyle=straightmove,
shortenstart=0.1em,
linewidth=3pt,
markmoves={g5-f6},
]

\hspace{0.35cm} XXXXXXXXXXXXXXXXXXXXXXX.
\end{minipage}

\begin{minipage}{0.45\textwidth}
\end{minipage}
\hfill
\begin{minipage}{0.45\textwidth}
\end{minipage}

\subsection{Exercice 33} %%%%%%%%%%%%%%%%%%%%%%%%%%%%%%%%%%%%%%%%%%%%

\newgame
\restoregame{strategie33}
\begin{minipage}{0.45\textwidth}
\hspace{0.7cm} Les noirs souhaite placer leur cavalier sur f5 pour attaquer le pion d4
\vspace{0.5cm}

\hspace{0.7cm} Si les blancs jouent le petit roque, les noir viennent attaquer le pion d4 qui est difficile à défendre sans risquer de perdre la paire de fous. 
% \mainline{1. O-O Nf5 2. Be3}
\vspace{0.5cm}

\hspace{0.7cm} Ici, les blancs ont un bon coup pour contrer le plan des noirs.
\vspace{0.5cm}

\end{minipage}
\hfill
\begin{minipage}{0.45\textwidth}
\chessboard[pgfstyle=color,
opacity=0.3,
color=green,
markfield=g8,
markfield=h6,
color=red,
markfield=f5,
]
\end{minipage}
%%%%%%%%%%%%%%%%%%%%%%%%%%%%%%%%%%%%%%%%%%%%%%%%%%%%%%%%%

%\begin{itemize}[leftmargin=1cm, label=\ding{32}, itemsep=1pt]
%\item {\bf } : \end{itemize}

%%%%%%%%%%%%%%%%%%%%%%%%%%%%%%%%%%%%%%%%%%%%%%%%%%%%%%%
