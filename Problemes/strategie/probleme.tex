
\newpage

\section{Énoncé des exercices}

Dans les problèmes suivant, l'analyse de la position permet de trouver un bon coup stratégique.

\subsection{Exercice 25}%%%%%%%%%%%%%%%%%%%%%%%%%%%%%%%%%%%%%%%%%%%%

\newgame
\fenboard{r4rk1/2p3b1/3p3p/p1pPp1p1/2P5/5PN1/PP4PP/1R3RK1 b − − 0 1}
\storegame{strategie25}
\begin{minipage}{0.45\textwidth}
\hspace{0.7cm}Il y a un bon coup stratégique pour les noirs.
\vspace{0.5cm}

\end{minipage}
\hfill
\begin{minipage}{0.45\textwidth}
\chessboard[
%showmover=false,
inverse,markstyle=leftborder,
]%\showboard
\end{minipage}

\subsection{Exercices 28} %%%%%%%%%%%%%%%%%%%%%%%%%%%%%%%%%%%%%%%%%%%%
\subsubsection{Exercice a} %%%%%%%%%%%%%%%%%%%%%%%%%%%%%%%%%%%%%%%%%%%%

\newgame
\fenboard{rnbqk2r/pp3ppp/2p2n2/3p2B1/1b1P4/2NB1Q2/PPP2PPP/R3K1NR b − − 0 1}
\storegame{strategie28a}
\begin{minipage}{0.45\textwidth}
\hspace{0.7cm} Ici, il y a plusieurs bon coup pour les noirs.
\vspace{0.5cm}

\hspace{0.7cm} Que joueriez-vous ?
\vspace{0.5cm}

\hspace{0.7cm} Quelle est la menace stratégique des blancs ? % Comment l'empêcher ?
\vspace{0.5cm}

\end{minipage}
\hfill
\begin{minipage}{0.45\textwidth}
\chessboard[
%showmover=false,
inverse,markstyle=leftborder,
]%\showboard
\end{minipage}

%\item 28b %%%%%%%%%%%%%%%%%%%%%%%%%%%%%%%%%%%%%%%%%%%%\subsection{28b}
\subsubsection{Exercice b} %%%%%%%%%%%%%%%%%%%%%%%%%%%%%%%%%%%%%%%%%%%%

\newgame
\fenboard{rn2k2r/pp3p2/2p1bp1p/3p4/1b1P4/P1NB1N2/1PP2PPP/R3K2R b − − 0 1}
\storegame{strategie28b}
\begin{minipage}{0.45\textwidth}
\hspace{0.7cm} Les noirs peuvent, soit échanger leur fou avec le cavalier, soit conserver leur fou.
\vspace{0.5cm}

\hspace{0.7cm} Quel est le meilleur choix ?
\vspace{0.5cm}

\end{minipage}
\hfill
\begin{minipage}{0.45\textwidth}
\chessboard[
%showmover=false,
inverse,markstyle=leftborder,
]%\showboard
\end{minipage}


%\item 28c %%%%%%%%%%%%%%%%%%%%%%%%%%%%%%%%%%%%%%%%%%%%\subsection{28c}
\subsubsection{Exercice c} %%%%%%%%%%%%%%%%%%%%%%%%%%%%%%%%%%%%%%%%%%%%

\newgame
\fenboard{r3k2r/pp1n1p2/2p1bp1p/3p4/3P4/P1PB1N2/2P2PPP/R3K2R b − − 0 1}
\storegame{strategie28c}
\begin{minipage}{0.45\textwidth}
\hspace{0.7cm}Ici, il y a un coup difficile à trouver pour les blancs. C'est un coup avec des idées, un plan.
\vspace{0.5cm}

\hspace{0.7cm} Ou se trouve la faiblesse des noirs ? Ou se trouve la faiblesse des blancs ? 
\vspace{0.5cm}

\hspace{0.7cm} Comment attaquer ces faiblesses ? Comment les défendre ? 
\vspace{0.5cm}

\end{minipage}
\hfill
\begin{minipage}{0.45\textwidth}
\chessboard
\end{minipage}

%\item 33 %%%%%%%%%%%%%%%%%%%%%%%%%%%%%%%%%%%%%%%%%%%%%%%%\subsection{33}
\subsection{Exercice 33} %%%%%%%%%%%%%%%%%%%%%%%%%%%%%%%%%%%%%%%%%%%%

\newgame
\fenboard{rn1qk2r/ppp1ppbp/3p2pn/8/2PP4/2N2P2/PP2BPPP/R1BQK2R w − − 0 1}
\storegame{strategie33}
\begin{minipage}{0.45\textwidth}
\hspace{0.7cm} Les noirs viennent de sortir leur cavalier. Quel est leur plan ?
\vspace{0.5cm}

\hspace{0.7cm} Les blancs jouent un coup qui stope le plan des noirs.
\vspace{0.5cm}

\end{minipage}
\hfill
\begin{minipage}{0.45\textwidth}
\chessboard[pgfstyle=color,
opacity=0.3,
color=green,
markfield=g8,
markfield=h6,
]
\end{minipage}
%\end{itemize}
%%%%%%%%%%%%%%%%%%%%%%%%%%%%%%%%%%%%%%%%%%%%%%%%%%%%%%%%%
%\begin{itemize}[leftmargin=0.7cm, itemsep=0pt]
%\item  \end{itemize}
