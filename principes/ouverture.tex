
\section{Ouvertures}
%
Stratégiquement, l'ouverture consiste à

\begin{itemize}[leftmargin=2.7cm, label=\ding{32}, itemsep=0pt]%\end{itemize}
\item  {\bf Se dévelloper}
\item  {\bf Controler le centre}
\item  {\bf Roquer}
\end{itemize}

Au cours de l'ouverture, il est souvent interessant de
\begin{itemize}[leftmargin=2.7cm, label=\ding{32}, itemsep=0pt]%\end{itemize}
\item  {\bf Gagner des  temps de développement}
\item  {\bf Faire jouer ses pièces ensemble / les protéger}
\end{itemize}

%%%%%%%%%%%%%%%%%%%%%
\subsection{Prendre le centre}
%%%%%%%%%%%%%%%%%%%%%
%
L'occupation du centre consiste à placer une pièce au centre. Cela peut être un pion ou une pièce mineure. Contrôler le centre consiste à placer une pièce qui attaque une ou plusieurs case du centre.


%%%%%%%%%%%%%%%%%%%%%
\subsection{Se développer}
%%%%%%%%%%%%%%%%%%%%%
%
Le développement consiste à sortir ses pièces mineures. Il faut éviter de jouer deux fois la même pièce dans cette phase.

%Parties d-échecs pédagogiques avec progression 🎓 1300 elo.mp4 : {\it Cf3 d4 d5 Cf6 Ce5}

%%%%%%%%%%%%%%%%%%%%%
\subsection{Roquer}
%%%%%%%%%%%%%%%%%%%%%
%
Le roque permet de mettre le roi à l'abri mais également de mettre les tours en liaison. C'est lorsque les tours sont en liaison que l'ouverture est terminé.


%%%%%%%%%%%%%%%%%%%%%
\subsection{Gagner des  temps de développement}
%%%%%%%%%%%%%%%%%%%%%
Créer une menace en se développant oblige l'adversaire à contrer cette menace, et permet de gagner du temps. Obliger l'adversaire à jouer deux fois la même pièce permet de retarder son développement.

Avancer un pion centrale pour attaqer un cavalier est souvent un bon coup.

%%%%%%%%%%%%%%%%%%%%%
\subsection{Faire jouer ses pièces ensemble}
%%%%%%%%%%%%%%%%%%%%%

Les pièces doivent se protéger entre elles et attaquer une même case faible de l'adversaire.

%%%%%%%%%%%%%%%%%%%%%%%%%%%%%%%%%%%%%%%%%%%%%%%%%%%%%%%%%%%%%%%%%%%%%%%%


%%%%%%%%%%%%%%%%%%%%%%%%%%%%%%%%%%%%%%%%%%%%%%%%%%%%%%%%%%%%%%%%%%%%%%%%
