
%%%%%%%%%%%%%%%%%%%%%
\chapter{Aide aux problèmes}
%%%%%%%%%%%%%%%%%%%%%


%%%%%%%%%%%%%%%%%%%%%
\section{Milieux}
%%%%%%%%%%%%%%%%%%%%%

\subsection{Sratégie}

\begin{minipage}{0.45\textwidth}
\hspace{0.7cm}Le centre est fermé, l'action va donc avoir lieu aux ailes. La position fermé donne de la force aux cavaliers, les fous ont peu d'espace.
\vspace{0.15cm}

\hspace{0.7cm}Les roques sont opposés, il faut donc attaquer rapidement le roque adverse (soit en envoyant les pions soit en amenant des pièces).

\vspace{0.15cm}
\hspace{0.7cm}Les noirs ont deux pions arriérés, les blanc ont des pions doublés

\vspace{0.15cm}
\hspace{0.7cm}Ou se trouve les cases faibles des noirs ? Comment y placer des pièces ?

%\vspace{0.5cm}
%\hspace{0.7cm}{\footnotesize source : youtube, chesstrainer2000.}
%\begin{itemize}[leftmargin=0.7cm, itemsep=0pt]
%\item  \end{itemize}
\end{minipage}
\hfill
\begin{minipage}{0.45\textwidth}
\newgame
\restoregame{milieuStrategie01}
%\showboard
\chessboard
\end{minipage}


%%%%%%%%%%%%%%%%%%%%%%%%%%%%%%%%%%%%%%%%%%%%%%%%%%%%%%%%%

\section{Finales}

\begin{minipage}{0.45\textwidth}
\hspace{0.5cm} Il existe trois types de sacrifices :

\begin{itemize}[leftmargin=1cm, label=\ding{32}, itemsep=1pt]
\item {\bf d'interception} : La prise va intercepter l'action d'une pièce adverse.
\item {\bf d'obstruction} : La pièce sacrifié obstrue l'action d'une pièce adverse.
\item {\bf de déviation} : La prise va dévier une pièce adverse.
\end{itemize}

\hspace{0.5cm} Ici, le bon coup est un sacrifice qui est à la fois d'interception, d'obstruction et de déviation...
\vspace{0.5cm}
\hspace{0.7cm}
%\begin{itemize}[leftmargin=0.7cm, itemsep=0pt]
%\item  \end{itemize}
\end{minipage}
\hfill
\begin{minipage}{0.45\textwidth}
\newgame
\restoregame{finaleTour01}
\chessboard
\end{minipage}




%\begin{itemize}[leftmargin=1cm, label=\ding{32}, itemsep=1pt]
%\item {\bf } : \end{itemize}

%%%%%%%%%%%%%%%%%%%%%%%%%%%%%%%%%%%%%%%%%%%%%%%%%%%%%%%
