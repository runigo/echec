\chapter{Principes stratégiques}

%Ce chapitre recense les principes stratégiques.

{\bf Controler des cases et y mettre une pièce dès que possible.}

%%%%%%%%%%%%%%%%%%%%%%%%%%%%%%%%%%
%\section{Ouverture}
%%%%%%%%%%%%%%%%%%%%%%%%%%%%%%%%%%

\begin{itemize}[leftmargin=2.7cm, label=\ding{32}, itemsep=0pt]%\end{itemize}
\item  {\bf Se dévelloper}
\item  {\bf Controler le centre}
\item  {\bf Roquer}
\end{itemize}

\begin{itemize}[leftmargin=2.7cm, label=\ding{32}, itemsep=0pt]%\end{itemize}
\item  {\bf Gagner des  temps de développement}
\item  {\bf Faire jouer ses pièces ensemble / les protéger}
\end{itemize}

%%%%%%%%%%%%%%%%%%%%%%%%%%
%\section{Milieu de partie}
%%%%%%%%%%%%%%%%%%%%%%%%%%
\begin{itemize}[leftmargin=2.7cm, label=\ding{32}, itemsep=0pt]%\end{itemize}
\item  {\bf Repérer les faiblesses}
\item  {\bf Reconnaître les schémas tactiques}
\item  {\bf Considérer les échanges de même valeur}
\item  {\bf Évaluer les finales}
\end{itemize}

\begin{itemize}[leftmargin=1.7cm, label=\ding{32}, itemsep=0pt]%\end{itemize}
\item  Mettre une tour sur la 7$^\text{ème}$ rangée dès que cela est possible.
\item  Doubler les tours sur la 7$^\text{ème}$ rangée dès que cela est possible.
\end{itemize}


%%%%%%%%%%%%%%%%%%%%%%%%%%%%%%%%%%
%\section{Exemple de plan}
%%%%%%%%%%%%%%%%%%%%%%%%%%%%%%%%%%

\begin{itemize}[leftmargin=2.7cm, label=\ding{32}, itemsep=0pt]%\end{itemize}
\item  {\bf Pièges à l'ouverture}
\item  {\bf Attaque de mat}
\item  {\bf Attaque d'une faiblesse}
\end{itemize}

\begin{itemize}[leftmargin=2.7cm, label=\ding{32}, itemsep=0pt]%\end{itemize}
\item  {\bf Défense} : contrer le plan adverse
\item  {\bf Consolidation} : solidifier une faiblesse
\end{itemize}

%%%%%%%%%%%%%%%%%%%%%%%%%%%%%%%%%%
%\section{Finale}
%%%%%%%%%%%%%%%%%%%%%%%%%%%%%%%%%%

Jouer des finales permet de s'améliorer et donne une meilleure compréhension du jeu.

Le roi devient une pièce trés forte, il faut l'activer.

%%%%%%%%%%%%%%%%%%%%%%%%%%%%%%%%%%%%%%%%%%%%%%%%%%%%%%%%%%%%%%%%%%%%%%%%
