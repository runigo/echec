

%\input{./utilisation/exemple.tex}

\section{Milieu de partie}
%
%
Au cours du milieux de partie,

\begin{itemize}[leftmargin=2.7cm, label=\ding{32}, itemsep=0pt]%\end{itemize}
\item  {\bf Repérer les faiblesses}
\item  {\bf Reconnaître les schémas tactiques}
\item  {\bf Considérer les échanges de même valeur}
\item  {\bf Évaluer les finales}
\end{itemize}

Mettre une tour sur la 7$^\text{ème}$ rangée dès que cela est possible.
Doubler les tours sur la 7$^\text{ème}$ rangée dès que cela est possible.

%%%%%%%%%%%%%%%%%%%%%
\subsection{Élaboration d'un plan}
%%%%%%%%%%%%%%%%%%%%%

Il est important de jouer des coups qui on du {\it sens}. Pour cela, il doivent suivre un plan, par exemple : 

\begin{itemize}[leftmargin=2.7cm, label=\ding{32}, itemsep=0pt]%\end{itemize}
\item  {\bf Pièges à l'ouverture}
\item  {\bf Attaque de mat}
\item  {\bf Attaque d'une faiblesse}
\item  {\bf Défense, consolidation}
\end{itemize}

%%%%%%%%%%%%%%%%%%%%%
%\subsection{Repérer les faiblesses}
%%%%%%%%%%%%%%%%%%%%%
%
%%%%%%%%%%%%%%%%%%%%%
%\subsection{Reconnaître les schémas tactiques}
%%%%%%%%%%%%%%%%%%%%%
%
%%%%%%%%%%%%%%%%%%%%%
%\subsection{Considérer les échanges de même valeur}
%%%%%%%%%%%%%%%%%%%%%
%
%%%%%%%%%%%%%%%%%%%%%
%\subsection{Évaluer les finales}
%%%%%%%%%%%%%%%%%%%%%

%%%%%%%%%%%%%%%%%%%%%%%%%%%%%%%%%%%%%%%%%%%%%%%%%%%%%%%%%%%%%%%%%%%%%%%%


%%%%%%%%%%%%%%%%%%%%%%%%%%%%%%%%%%%%%%%%%%%%%%%%%%%%%%%%%%%%%%%%%%%%%%%%
