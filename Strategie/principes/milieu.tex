

%\input{./utilisation/exemple.tex}

\section{Milieu de partie}

\begin{itemize}[leftmargin=4.7cm, label=\ding{32}, itemsep=0pt]%\end{itemize}
\item  {\bf Repérer les faiblesses}
\item  {\bf Reconnaître les schémas tactiques}
\item  {\bf Considérer les échanges de même valeur}
\item  {\bf Évaluer les finales}
\end{itemize}

\begin{itemize}[leftmargin=2.7cm, label=\ding{32}, itemsep=0pt]%\end{itemize}
\item  { Contrôler des cases et y mettre une pièce dès que possible.}
\item  { Mettre une tour sur la 7$^\text{ème}$ traverse dès que cela est possible.}
\item  { Doubler les tours sur la 7$^\text{ème}$ traverse dès que cela est possible.}
\end{itemize}

\subsection{Proverbes échiquéens}

\begin{itemize}[leftmargin=2.7cm, label=\ding{32}, itemsep=0pt]%\end{itemize}
\item  {\bf Devant les exceptions, les principes doivent s'effacer.}
\item  {\bf Qui possède les fous possède l'avenir.}
\item  {\bf C'est avec les pièces fortes qu'il faut occuper les positions fortes.}
\item  {\bf La menace est plus forte que son exécution.}
\end{itemize}

%%%%%%%%%%%%%%%%%%%%%
\subsection{Élaboration d'un plan}
%%%%%%%%%%%%%%%%%%%%%

Il est important de jouer des coups qui ont du sens. Pour cela, ils doivent suivre un plan, par exemple : 

\begin{itemize}[leftmargin=2.7cm, label=\ding{32}, itemsep=0pt]%\end{itemize}
\item  {\bf Pièges à l'ouverture}
\item  {\bf Attaque de mat}
\item  {\bf Attaque d'une faiblesse}
\item  {\bf Défense, consolidation}
\end{itemize}

L'adversaire suit son plan, il est important de détecter ce plan afin de le contrer.

{\footnotesize %XIX$^\text{e}$ siècle — {\it }

Si on subit une attaque sur l'aile roi (après le petit roque), l'échange de pièces permet de diminuer la pression de l'attaque.

Si on est attaqué au centre, envisager de contre-attaquer à l'aile, si on est attaqué à l'aile, envisager de contre-attaquer au centre.


}

%%%%%%%%%%%%%%%%%%%%%
\subsection{Ne pas prendre les pièces clouées}
%%%%%%%%%%%%%%%%%%%%%

{\bf La menace est plus forte que son exécution.}

Un clouage met dans une position embarrassante. La prise de la pièce clouée supprime cette situation embarrassante.

Si l'on n'est pas obligé de prendre, il vaut mieux conserver le clouage et chercher à augmenter la pression (créer une autre menace, attaquer une nouvelle fois la pièce clouée, ...).


%%%%%%%%%%%%%%%%%%%%%
%\subsection{Repérer les faiblesses}
%%%%%%%%%%%%%%%%%%%%%
%
%%%%%%%%%%%%%%%%%%%%%
%\subsection{Reconnaître les schémas tactiques}
%%%%%%%%%%%%%%%%%%%%%
%
%%%%%%%%%%%%%%%%%%%%%
\subsection{Considérer les échanges de même valeur}
%%%%%%%%%%%%%%%%%%%%%

Un cavalier bien placé peut être beaucoup plus fort qu'un fou ayant peu d'espace ou peu de cible. L'échange d'un fou peu actif contre un bon cavalier est généralement avantageux.

Un cavalier très bien placé peut être beaucoup plus fort qu'une tour. Lorsqu'on subit une forte attaque, échanger une tour contre un cavalier peut être avantageux.

Un fou ayant de l'espace peut être très fort, un cavalier mal placé peut être très faible.

%Stratégiquement, il faut éviter d'échanger trop rapidement la paire de fou dans l'ouverture.



%%%%%%%%%%%%%%%%%%%%%
%\subsection{Évaluer les finales}
%%%%%%%%%%%%%%%%%%%%%

%%%%%%%%%%%%%%%%%%%%%%%%%%%%%%%%%%%%%%%%%%%%%%%%%%%%%%%%%%%%%%%%%%%%%%%%


%%%%%%%%%%%%%%%%%%%%%%%%%%%%%%%%%%%%%%%%%%%%%%%%%%%%%%%%%%%%%%%%%%%%%%%%
