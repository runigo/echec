
%%%%%%%%%%%%%%%%%%%%%
\section{Élaboration d'un plan}
%%%%%%%%%%%%%%%%%%%%%

À la sortie de l'ouverture, nos pièces légères sont sorties, le roque à été effectué, la dame s'est dévellopée et les tours en liaison se sont placées sur les bonnes colonnes (ouvertes, semi-ouvertes, en face de la dame adverse). On entame alors le milieu de partie, et souvent, on arrive alors à la position critique. La position critique, c'est la position ou l'on ne sait plus quoi jouer. Souvent dans cette position, on fait un mauvais coup et l'adversaire en profite. Il s'agit donc de la position où l'on doit réfléchir un peu et trouver un plan.

Regarder la position de l'adversaire, détecter ses faiblesses.

Questions à se poser :

Où mes pièces seraient bien placées ? Détecter des cases faibles chez l'adversaire, quel type de pièce serait forte sur ces cases, comment ammener notre pièce sur cette cases.

Quel est ma pièce la plus mal placée ? Où serait-elle bien ?


%%%%%%%%%%%%%%%%%%%%%
\subsection{Exemples de plan}
%%%%%%%%%%%%%%%%%%%%%



\subsubsection{Exploitation d'une case faible}

On détecte une case faible dans le camp adverse et notre cavalier y serait très fort. Pour atteindre cette case,  il doit emprunter une case controlée par la dame adverse. Un de nos pion peut être poussé afin de controler cette case. Tout cela constitue un plan, 3 ou 4 coups peuvent être nécessaire à son exécution. Pendant ce temps, il faut aussi réagir aux coups de l'adversaire, contrer une menace, protéger une pièce attaquée, ... Ainsi le plan initial peut être réalisé en 8 ou 10 coups, pendant ces 8 ou 10 coups, on joue des coups en vue du plan, on sait quoi jouer. Enfin, si on y arrive, on se retrouve avec un cavalier bien placé, ce qui est un avantage.

\subsubsection{Exploitation d'un pion faible}




%%%%%%%%%%%%%%%%%%%%%%%%%%%%%%%%%%%%%%%%%%%%%%%%%%%%%%%%%%
