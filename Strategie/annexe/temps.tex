
%%%%%%%%%%%%%%%%%%%%%
\chapter{Le temps}
%%%%%%%%%%%%%%%%%%%%%

C'est le nombres de mouvement de pièces pour atteindre la position. Les mouvements de pions n'apporte pas de temps.

Le roque vaut un temps sauf si les deux camps ont roqué. Les cavaliers valent
1 temps sur la 2$^\text{nde}$ et la 3$^\text{ème}$ traverse, 2 temps sur la
4$^\text{ème}$ et 5$^\text{ème}$, 3 temps sur la 6$^\text{ème}$ ou la
7$^\text{ème}$.

Une tour vaut un temps sur la première traverse si elle est développée, si elle est active ( si elle a de l'espace vers des bonnes cases).
%\begin{itemize}[leftmargin=1cm, label=\ding{32}, itemsep=1pt]
%\item {\bf } : \end{itemize}


\section{Exemple}
%%%%%%%%%%%%%%%%%%%%%%%%%%%%%


%\subsection{exemple}

\newgame
\begin{minipage}{0.5\textwidth}
\fenboard{r5rk/1ppqb1pp/2bpNp1n/p4P2/2Q1P2B/2N5/PPP3PP/3R1RK1 w − − 0 20}
%\showboard
\chessboard
\end{minipage}
\begin{minipage}{0.5\textwidth}
Pour les blancs : le cavalier e6 vaut trois temps car il est sur la 6$^\text{éme}$ traverse, la tour f1 vaut 1 temps car elle a de l'espace. Ajouté à cela la tour d1, la dame c4, le fou h4 et le cavalier c3 Les blancs ont donc 8 temps.

Pour les noirs : la tour g8 n'a pas d'espace, la tour a8 n'est pas développée. Les fou, le cavalier et la dame donne 4 temps.
\end{minipage}

Les blancs ont donc ici un net avantage de temps (4 temps d'avance). En contre-partie, ils disposent d'un pion arriéré.


%\section{Le temps}
%%%%%%%%%%%%%%%%%%%%%%%%%%%%%
%\fenboard{r5k1/1b1p1ppp/p7/1p1Q4/2p1r3/PP4Pq/BBP2b1P/R4R1K w − − 0 20}
%\mbox{}\bigskip\showboard
%\mainline{20.Qxb7 Rae8 21.Qd5}\showboard
%\fenboard{r5k1/1b1p1ppp/p7/1p1Q4/2p1r3/PP4Pq/BBP2b1P/R4R1K w − − 0 20}

%\mbox{}
%\bigskip

%\showboard







%%%%%%%%%%%%%%%%%%%%%%%%%%%%%%%%%%%%%%%%%%%%%%%%%%%%%%%
