%%%%%%%%%%%%%%%%%%%%%%%%%%%%%%%%%%%%%%%%%%%%%%%%%%%%%%%%%%
%
\section{La partie espagnole}
%
%%%%%%%%%%%%%%%%%%%%%%%%%%%%%%%%%%%%%%%%%%%%%%%%%%%%%%%%%%
%
\newgame
\begin{minipage}{0.5\textwidth}
\mainline{1. e4 e5 2. Nf3 Nc6 3. Bb5 }

\chessboard
\storegame{espagnole}
\end{minipage}
\begin{minipage}{0.5\textwidth}
\hspace{0.7cm} C'est la suite la plus forte et la plus consistante du jeu du cavalier du roi. Il existe deux bonnes défenses : la défense berlinoise (ligne naturelle \mainline{3... Nf6}) et la {\it ligne principale} \restoregame{espagnole}
\mainline{3... a6 4. Bb5 Nf6}

\end{minipage}


\subsection{Ligne principale}
%%%%%%%%%%%%%%%%%%%%%%%%%%%%%%
%\newgame
\begin{minipage}{0.5\textwidth}
\restoregame{espagnole}
\mainline{3... a6 }

\chessboard
%\storegame{espagnole}
\end{minipage}
\begin{minipage}{0.5\textwidth}
\mainline{4. Ba4 Nf6}% 5.O-O Ne4

\chessboard
\end{minipage}


\subsection{Ligne naturelle}
%%%%%%%%%%%%%%%%%%%%%%%%%%%%%%
%\newgame
\begin{minipage}{0.5\textwidth}
\restoregame{espagnole}
\mainline{3... Nf6}

\chessboard
%\storegame{espagnoleNaturelle}
\end{minipage}
\begin{minipage}{0.5\textwidth}
\mainline{4.O-O Ne4}

\chessboard
\end{minipage}

%
%%%%%%%%%%%%%%%%%%%%%%%%%%%%%%%%%%%%%%%%%%%%%%%%%%%%%%%%%
