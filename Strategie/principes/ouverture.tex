
\section{Ouvertures}
%
%L'ouverture consiste à

\begin{itemize}[leftmargin=4.7cm, label=\ding{32}, itemsep=0pt]%\end{itemize}
\item  {\bf Se développer}
\item  {\bf Contrôler le centre}
\item  {\bf Roquer}
%\item  {\bf Gagner des  temps de développement}
%\item  {\bf Faire jouer ses pièces ensemble / les protéger}
\end{itemize}

Il s'agit de la première bataille :  prendre le centre et empêcher l'adversaire de le prendre, gagner des temps de développement, créer des faiblesses chez l'adversaire et éviter de s'en créer.%, ...

%Un plan peut être de laisser l'adversaire s'installer au centre pour ensuite attaquer ce centre.

%%%%%%%%%%%%%%%%%%%%%
\subsection{Prendre le centre}
%%%%%%%%%%%%%%%%%%%%%
%
%L'occupation du centre consiste à placer une pièce au centre. Cela peut être un pion ou une pièce mineure. Contrôler le centre consiste à placer une pièce qui attaque une ou plusieurs case du centre.


\begin{minipage}{0.45\textwidth}
\hspace{0.5cm} Dans la plupart des parties, les blancs jouent 1.e4 et menacent 2.d4 ou jouent 1.d4 et menacent 2.e4. Le premier plan des noirs est de contrer la menace,
\begin{itemize}[leftmargin=0.7cm, itemsep=0pt]%\end{itemize}
\item  soit en opposant un pion,
\item  soit en attaquant le pion des blancs.
%\item  {\bf Faire jouer ses pièces ensemble / les protéger}
\end{itemize}
\hspace{0.5cm} Un autre plan des noirs peut être de se préparer à attaquer ultérieurement le fort centre des blancs.
\end{minipage}
\hfill
\begin{minipage}{0.45\textwidth}
\newgame
\fenboard{8/8/8/8/3PP3/8/8/8 w − − 0 20}
%\showboard
\chessboard
\end{minipage}

%%%%%%%%%%%%%%%%%%%%%
\subsection{Se développer}
%%%%%%%%%%%%%%%%%%%%%
% sortir les pièces légère (cavaliers et fous)
Le développement consiste à sortir ses pièces légère (cavaliers et fous). Il faut éviter de jouer deux fois la même pièce dans cette phase. La sortie des pièces doit suivre un plan : protéger un pion attaqué, clouage, attaquer un pion adverse, ...

%
\newgame
\begin{minipage}{0.45\textwidth}

Exemple de la partie du cavalier du roi :
\begin{itemize}[leftmargin=0.7cm, itemsep=0pt]%\end{itemize}
\item  \mainline{1. e4 } : Prend le centre et menace d4.
\item  \mainline{1... e5} : Prend le centre et pare la menace.
\item  \mainline{2. Nf3 } : Développe le cavalier en attaquant le pion noir.
\item  \mainline{2... Nc6} : Défend le pion en développant une pièce.
\end{itemize}

\end{minipage}
\hfill
\begin{minipage}{0.45\textwidth}
%\hspace{0.7cm} C'est la suite la plus forte et la plus consistante du jeu du cavalier du roi. Il existe deux bonnes défenses : la défense berlinoise (ligne naturelle \mainline{3... Nf6}) et la ligne principale \restoregame{espagnole}
%\mainline{3... a6 4. Bb5 Nf6}

\chessboard
\storegame{cavalierRoi}

\end{minipage}

%Parties d-échecs pédagogiques avec progression 🎓 1300 elo.mp4 : {\it Cf3 d4 d5 Cf6 Ce5}

%%%%%%%%%%%%%%%%%%%%%
\subsection{Roquer}
%%%%%%%%%%%%%%%%%%%%%
%
{\bf Le roque permet de mettre le roi à l'abri et de mettre les tours en liaison.} %L'ouverture est terminé lorsqu'il ne reste plus que les deux tours
%C'est lorsque les tours sont en liaison que l.


%%%%%%%%%%%%%%%%%%%%%
\subsection{Gagner des temps de développement}
%%%%%%%%%%%%%%%%%%%%%
Créer une menace en se développant oblige l'adversaire à contrer cette menace, et permet de gagner du temps.
Obliger l'adversaire à jouer deux fois la même pièce permet de retarder son développement
(avancer un pion central pour attaquer un cavalier est souvent un bon coup).

%{\footnotesize Avancer un pion centrale pour attaquer un cavalier est souvent un bon coup.}

%%%%%%%%%%%%%%%%%%%%%
\subsection{Faire jouer ses pièces ensemble}
%%%%%%%%%%%%%%%%%%%%%

Les pièces doivent se protéger entre elles. Lorsqu'une de nos pièces n'est pas protégé, il faut la surveiller, lorsqu'une pièce adverse n'est pas protégée, on peut l'attaquer.



%%%%%%%%%%%%%%%%%%%%%%%%%%%%%%%%%%%%%%%%%%%%%%%%%%%%%%%%%%%%%%%%%%%%%%%%


%%%%%%%%%%%%%%%%%%%%%%%%%%%%%%%%%%%%%%%%%%%%%%%%%%%%%%%%%%%%%%%%%%%%%%%%
