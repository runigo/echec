
%%%%%%%%%%%%%%%%%%%%%
\chapter{Solutions des problèmes}
%%%%%%%%%%%%%%%%%%%%%



%%%%%%%%%%%%%%%%%%%%%
\section{Milieux}
%%%%%%%%%%%%%%%%%%%%%

\subsection{Sratégie}

\newgame
\restoregame{strategie25}
\begin{minipage}{0.45\textwidth}
\hspace{0.7cm}
\vspace{0.25cm}

\hspace{0.7cm} Il s'agit d'activer le fou grâce au sacrifice \mainline{1... e4}
\vspace{0.25cm}

\hspace{0.7cm} Cela permet au cavalier blanc de venir sur la bonne case e4 en gagnant un pion.
\vspace{0.25cm}

\hspace{0.7cm} En revanche, cela permet d'activer le mauvais fou des noirs et de rayonner sur les cases noires dans le camps des blancs. \mainline{2. Ne4 Bd4}
\vspace{0.25cm}

\end{minipage}
\hfill
\begin{minipage}{0.45\textwidth}
%\chessboard%\showboard
\chessboard[
%showmover=false,
inverse,markstyle=leftborder,
]%\showboard
\end{minipage}

Si les tours viennent à s'échanger, la finalle sera favorable au noir : 
{\bf En finalle fou contre cavalier, s'il y a des pions sur les deux ailes, le fou est meilleur.}

{\bf Il faut rendre la position favorable pour nos pièces légères.} Si on a un fou : 

\begin{itemize}[leftmargin=1cm, label=\ding{32}, itemsep=1pt]
\item  Placer ses pions sur la couleur opposée
\item  Ouvrir la position
\end{itemize}

\section{Finales}

\begin{minipage}{0.45\textwidth}
\newgame
\restoregame{strategie25}

%\hspace{0.3cm} Le bon coup est \mainline{1. Rd6}.

\begin{itemize}[leftmargin=1cm, label=\ding{32}, itemsep=1pt]
\item \hspace{0.7cm}La prise par le pion (c7xd6) {\bf intercepte} la protection du pion noir d2.
\item \hspace{0.7cm}La tour blanche {\bf obstrue} la défense du pion noir d2.
\item \hspace{0.7cm}La prise par la tour noir {\bf dévie} celle-ci de la huitième traverse, cela permet la promotion sur échec par g7-g8.
\end{itemize}

\vspace{0.5cm}
\hspace{0.7cm}
\end{minipage}
\hfill
\begin{minipage}{0.45\textwidth}
\chessboard[
%showmover=false,
inverse,markstyle=leftborder,
]%\showboard
\end{minipage}


%%%%%%%%%%%%%%%%%%%%%%%%%%%%%%%%%%%%%%%%%%%%%%%%%%%%%%%
%\begin{itemize}[leftmargin=1cm, label=\ding{32}, itemsep=1pt]
%\item {\bf } : \end{itemize}
