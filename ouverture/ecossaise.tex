%%%%%%%%%%%%%%%%%%%%%%%%%%%%%%%%%%%%%%%%%%%%%%%%%%%%%%%%%%
%
\section{Jouer l'écossaise}
%
%%%%%%%%%%%%%%%%%%%%%%%%%%%%%%%%%%%%%%%%%%%%%%%%%%%%%%%%%%
%
%L'écossaise se caractérise par le coup \texttt{3. d4} après les coups classiques \texttt{1. e4 e5 2. Cf3 Cc6} :
Après les coups classiques, fréquement joués
\newgame

\mainline{1. e4 e5 2. Nf3 Nc6}

\begin{center}
\chessboard
\end{center}


L'écossaise se caractérise par le coup
%les blancs rentre généralement dans la partie espagnole ou la partie italienne. Il peuvent également rentré dans la partie écossaise
%\textit{Projet Cavalier e5, Tour e1}
\mainline{3. d4}

%\textit{Fréquent : }\textit{}
%\variation{4. e5}


%\mainline{4... Bg4}

\begin{center}
\chessboard
\end{center}


\newgame

\mainline{1. e4 e6 2. d4 d5 3. e5}

\begin{center}
\chessboard
\end{center}


\newgame

\textit{Projet Cavalier e5, Tour e1}
\mainline{1. e4 e6 2. d4 d5 3. pxd5}

\begin{center}
\chessboard
\end{center}


\newgame
%rnbqkbnr/pppppppp/8/8/8/8/PPPPPPPP/RNBQKBNR

\begin{center}
\chessboard{rnbqkbnr/pppppppp/8/8/8/8/PPPPPPPP/RNBQKBNR w KQkq - 0 1}
\end{center}



%\mainline{3. exd}
%%%%%%%%%%%%%%%%%%%%%%%%%%%%%%%%%%%%%%%%%%%%%%%%%%%%%%%%%
