
%%%%%%%%%%%%%%%%%%%%%
\chapter{Solutions des problèmes}
%%%%%%%%%%%%%%%%%%%%%



%%%%%%%%%%%%%%%%%%%%%
\section{Milieux}
%%%%%%%%%%%%%%%%%%%%%

\subsection{Sratégie}

\newgame
\restoregame{milieuStrategie01}
\begin{minipage}{0.45\textwidth}
\hspace{0.7cm}La case faible est la case f5, un cavalier y serait très bien (attaque le pion arriéré d6 et le fou noir qui défend bien).
\vspace{0.25cm}

\hspace{0.7cm}Le cavalier des blancs doit participer à l'attaque. Pour arriver en f5, il peut passer par e3.

\vspace{0.25cm}
\hspace{0.7cm}Le bon coup est donc \mainline{1. Nd1} (avec le plan de continuer en e3 puis f5 et de rentrer une tour en h7 pour attaquer le fou des noirs)

%{\footnotesize source : youtube, chesstrainer2000.}
%\begin{itemize}[leftmargin=0.7cm, itemsep=0pt]
%\item  \end{itemize}
\end{minipage}
\hfill
\begin{minipage}{0.45\textwidth}
\chessboard[arrow=latex,
pgfstyle=straightmove,
%markmove=a1-a8,
%arrow=to,linewidth=0.2ex,
color=red,
%pgfstyle=knightmove,
pgfstyle={[linewidth=2pt,arrow=to,style=knight]curvemove},
pgfstyle=curvemove,
%markmoves={c3-d1},  % Comenté pour effacer la flèche
shortenstart=-1ex,
%markmoves=h1-g3
]
%\chessboard%\showboard
\end{minipage}



\section{Finales}

\begin{minipage}{0.45\textwidth}
\newgame
\restoregame{finaleTour01}

\hspace{0.3cm} Le bon coup est \mainline{1. Rd6}.

\begin{itemize}[leftmargin=1cm, label=\ding{32}, itemsep=1pt]
\item \hspace{0.7cm}La prise par le pion (c7xd6) {\bf intercepte} la protection du pion noir d2.
\item \hspace{0.7cm}La tour blanche {\bf obstrue} la défense du pion noir d2.
\item \hspace{0.7cm}La prise par la tour noir {\bf dévie} celle-ci de la huitième traverse, cela permet la promotion sur échec par g7-g8.
\end{itemize}

\vspace{0.5cm}
\hspace{0.7cm}
%\begin{itemize}[leftmargin=0.7cm, itemsep=0pt]
%\item  \end{itemize}
\end{minipage}
\hfill
\begin{minipage}{0.45\textwidth}
\chessboard
\end{minipage}


%\begin{itemize}[leftmargin=1cm, label=\ding{32}, itemsep=1pt]
%\item {\bf } : \end{itemize}

%%%%%%%%%%%%%%%%%%%%%%%%%%%%%%%%%%%%%%%%%%%%%%%%%%%%%%%
